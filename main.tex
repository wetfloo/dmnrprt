\documentclass[fontsize=14pt, paper=a4, pagesize, DIV=calc]{scrreprt}


\usepackage[
  left=3cm,
  right=1cm,
  top=2cm,
  bottom=2cm, 
  includeheadfoot,
  nohead,
  ]{geometry}


\usepackage{setspace}
\onehalfspacing


% \usepackage{ragged2e}


% \usepackage{polyglossia}
% \setdefaultlanguage{russian} %% устанавливает главный язык документа
% \setotherlanguage{english} %% устанавливает второй язык документа
% \defaultfontfeatures{Ligatures=TeX} %% задаёт свойства шрифтов по умолчанию
% \setmainfont{Times New Roman}
% \setsansfont{Arial} %% задаёт шрифт без засечек
% \setmonofont{Times New Roman} %% задаёт моноширинный шрифт
% doesn't work properly with lipsum, apparently


\usepackage[main=russian, english]{babel}
\babelfont{rm}{Times New Roman}
\babelfont{sf}{Arial}
\babelfont{tt}{FreeMono}

\usepackage{fontspec}
\defaultfontfeatures{Ligatures=TeX}
\newfontfamily\codefont[Scale=MatchLowercase]{Courier New} % for the love of god use fonts that actually look good


\usepackage{csquotes}

\setlength{\parindent}{5ex}


% getting that justfied w*rd look
% comment two lines below to get hyphenation back and get rid of infinite amount of overfull warnings
\usepackage[none]{hyphenat}
\sloppy


\usepackage{array} % using this go get a table going for the title page


\renewcaptionname{russian}{\contentsname}{\MakeUppercase{Содержание}}           %Table of contents
\renewcaptionname{russian}{\listfigurename}{Список рисунков}    %Figures
\renewcaptionname{russian}{\listtablename}{Таблицы}        %Tables
\renewcaptionname{russian}{\figurename}{Рисунок}               %Figure
\renewcaptionname{russian}{\tablename}{Таблица}                %Table


\setlength{\intextsep}{24pt}


\usepackage{chngcntr}
\counterwithout{section}{chapter}


\usepackage{titlesec}

\titleformat{\chapter}{\normalfont\Large\bfseries\filcenter}{}{0em}{}
\titlespacing*{\chapter}{0pt}{0pt}{24pt}

\titleformat{\section}{\normalfont\bfseries}{\thesection.~}{1em}{}
\titlespacing*{\section}{14pt}{24pt}{16pt}

\titleformat{\subsection}{\normalfont\bfseries}{\thesubsection.~}{1em}{}
\titlespacing*{\subsection}{14pt}{24pt}{16pt}

\titleformat{\subsubsection}{\normalfont\bfseries}{\thesubsubsection.~}{1em}{}
\titlespacing*{\subsubsection}{14pt}{24t}{16pt}


\renewcommand{\thefigure}{\arabic{figure}}

\usepackage{caption}
\DeclareCaptionLabelSeparator{dash}{ -- }% or $\vert$
\captionsetup{
  labelsep=dash,
}


\usepackage{titletoc}

\titlecontents{chapter}
[0em]                   % adjust left margin
{\rmfamily}             % font formatting
{}{}{\titlerule*[1pc]{.}\contentspage}


\usepackage{enumitem}
\setenumerate[1]{label={\arabic*)}} % Global setting


\usepackage{hyperref}
\hypersetup{hidelinks, linktoc=all}


\usepackage[
  backend=biber,
  bibencoding=utf8,
  style=gost-numeric,
  language=autobib,
  movenames=true,
  minnames = 1, 
  maxnames = 3, %сократит до первого автора с добавлением [и др.], если число авторов перевышает maxnames
  autocite=superscript,
  natbib=true,
]{biblatex}

\addbibresource{bibliography/a.bib}
%Приведение списка литературы к требованим СФУ
\renewcommand*{\mkgostheading}[1]{{#1}}
\toggletrue{bbx:gostbibliography}%
\renewcommand*{\revsdnamepunct}{\addcomma}%Добавление запятой после фамилии и перед инициалами
\renewcommand*{\labelnamepunct}{\addperiod\space}


\usepackage[language=english]{lipsum} % just to test if thing are working correctly


\urlstyle{same} % changing url font to the one we use for the rest of the text


\usepackage{graphicx}
\graphicspath{ {./pics/} }
\DeclareGraphicsExtensions{.pdf,.png,.jpg,.svg}

\usepackage{float} % used to make figures actually stay in the place we put them in
% \usepackage[section]{placeins}

\usepackage{scrhack} % stop the compiler from complaining

\usepackage{listings}
% \lstdefinestyle{sql}{
%   language=SQL,
%   basicstyle=\codefont,
% }
% \usepackage{fancyvrb}

\usepackage{indentfirst}

\begin{document}

\begin{titlepage}
  \newgeometry{top=10mm, bottom=10mm, left=10mm, right=10mm}
  \begin{center}
    Федеральное государственное автономное\\
    образовательное учреждение\\
    высшего образования\\
    <<СИБИРСКИЙ ФЕДЕРАЛЬНЫЙ УНИВЕРСИТЕТ>>\\
    Институт космических и информационных технологий\\
    \vfill
    {\Large\textbf{ОТЧЕТ О ПРАКТИКЕ}}\\
    Место прохождения праткики: ул. Шишкина\\
    Тема: кого вообще волнует
  \end{center}
  \vfill

  \begin{center}
    \begin{tabular}{l c m{0.01\textwidth} l}
      Преподаватель                                & \rule{3cm}{0.1pt} &  & Препод~Н.В. \\[5mm]
      Студент FF20-66в, \textnumero\ з/к 123456789 & \rule{3cm}{0.1pt} &  & Пупкин~В.И. \\
    \end{tabular}
  \end{center}

  \begin{center}
    Красноярск {\the\year}
  \end{center}
\end{titlepage}

\restoregeometry

\tableofcontents

\newpage

\chapter{ВВЕДЕНИЕ}

Одним из самых больших преимуществ Интернета является свобода, которую он предоставляет пользователям. Хотите транслировать фильм? Выбирайте из сотен библиотек. Никакой телевизор не выбирает, что смотреть за вас. В результате миллионы людей предпочитают смотреть видео и развлекательный контент в Интернете.

До того, как в середине 2000-х стало популярным онлайн-телевидение, было создано интернет-радио. Сегодня люди все чаще транслируют радиоконтент онлайн по многим причинам. Ниже приведены некоторые из них:

\begin{enumerate}
  \item \textbf{Простота доступа.}
        Доступ к онлайн-радиостанциям можно получить с любого устройства в любом месте, если у вас есть подключение к Интернету. Если вы едете на поезде на работу или в отпуск, вы можете получить доступ к музыкальным приложениям. Если у вас хорошее широкополосное соединение, нет никаких ограничений на то, что вы можете слушать.

        Большинство пользователей онлайн-радио получают доступ к станциям через свои мобильные устройства.
  \item \textbf{Удобство.}
        Онлайн-радио и подкасты предлагают безграничное удобство для современных слушателей. В нашем занятом мире люди хотят потреблять много контента одновременно. Они могут захотеть слушать свои любимые подкасты, например, во время обеденного перерыва на работе. Они хотят сохранять новые песни и слушать их потом.

        Ничто не предлагает лучшего удобства для потребления радиоконтента, чем онлайн-радио. Будь то подкасты или музыкальные приложения, контролировать, что потреблять и когда это делать, можно одним щелчком значка. Переключаться между песнями и сохранять избранные для последующего прослушивания стало проще, чем когда-либо прежде.
  \item \textbf{Разнообразие.}
        Лучшие приложения для онлайн-радио предлагают больше музыки, чем у вас когда-либо было время. Подобно традиционным радио, музыкальные приложения позволяют исполнителям предлагать свои песни. Но, в отличие от обычных радиостанций, планка того, какую музыку можно публиковать в Интернете, невысока.

        Любой, у кого есть голос и записывающее устройство, может записывать свое пение и отправлять песни в музыкальные приложения. Конечно, вам потребуется немного усилий, чтобы произвести впечатление на слушателей.

        Благодаря большому разнообразию песен и исчерпывающему списку категорий все больше людей отказываются от традиционных радиостанций в пользу онлайн-аналогов. С Интернетом не имеет значения, насколько уникален ваш музыкальный вкус. Где-то на онлайн-радиостанции кто-то играет.
\end{enumerate}
 % useless info about internet radio to fill up space

\chapter{Цель работы и задачи}

\textbf{Цель работы:} разработать интернет-радио площадку с возможностью круглосуточно вещать композиции из плей-листов, а также вести прямой эфир\\


\textbf{Задачи: }
\begin{enumerate}
  \item Изучить существующие варианты и выбрать подходящий
  \item Установить выбранное интернет-радио на локальный сервер
  \item Установить выбранное интернет-радио на реальный хостинг
\end{enumerate}
\chapter{Ход работы}

\section{Выбор програмного обеспечения для трансляции аудио}

Для создания интернет радиостанции были рассмотрены следующие варианты програмного обеспечения:

\begin{itemize}
  \item \textbf{Icecast}
        -- это сервер потокового мультимедиа (аудио/видео),  который в поддерживает потоки Ogg (Vorbis и Theora), Opus, WebM и MP3.

        Его можно использовать для создания интернет-радиостанции или частного музыкального автомата и многого другого. Он очень универсален в том смысле, что новые форматы могут быть добавлены относительно легко, и поддерживает открытые стандарты для общения и взаимодействия. Однако, его довольно трудно использовать сам по себе.
  \item \textbf{Media Server Control Panel}
        -- это самая передовая панель управления хостингом интернет-радио/телевидения с широким функционалом. Это автономное приложение, предоставляющее собственные веб- и FTP-сервисы. Однако, оно платно и имеет закрытый исходный код.
  \item \textbf{AzuraCast}
        -- это самодостаточный универсальный пакет для управления веб-радио с открытым исходным кодом. Он позволяет загружать мультимедиа, управляйтб плейлистами, создавать локальные точки монтирования и удалённые ретрансляторы, просматривать аналитику и отчёты из веб-браузера. Также AzuraCast легко устанавливается на сервер при помощи Docker-образа на сайте. Именно это ПО и было выбрано.
\end{itemize}

\section{Установка, настройка и тестирование интернет-радио на локальном сервере}

Следующим шагом после выбора программного обеспечения AzuraCast было изучать документации с оффициального сайта. Следуя официальному руководству, была установлена виртуальная машина Ubuntu for Windows и произведена установка AzuraCast на неё.

После перехода по стандартному IP - 127.0.0.1, мы попадаем на окно первоначальной настройки. \\
Создаём пользователя. \\
Создаём радиостанцию. \\
Вписываем URL-адрес сайта. \\
После этого мы попадаем на окно управления самой радиостанцией. Дальше переходим в медиафайлы. \\
Добавляем треки в хранилище сайта и добавляем их в плейлист default. \\
После этого мы можем запустить станцию и перейти по адресу 127.0.0.1/public/teststation, чтобы протестировать работу.
\section{Установка, настройка и тестирование интернет-радио на реальном хостинге}

Как только мы удостоверились, что наше интернет-радио работает стабильно и без ошибок, можем переходить к загрузке его на хостинг.

Был выбран VPS хостинг под управлением операционной системы Ubuntu Linux. После оплаты аренды, были получены данные для подключения к терминалу Linux по протоколу SSH.

Дальше, с помощью программного обеспечения Bitvise SSH Client, было произведено подключение к терминалу, и проделаны действия, аналогичные тем, что и при установке интернет-радио на локальном сервере.

По итогу, после настройки интернет-радио, мы получили и протестировали следующий результат.


\chapter{Заключение}

В результате проделанной работы нами были получены как теоретические, так и практические знания, относящиеся как к интернет-радио, так и работе с VPS-хостингами. Итогом работы можно назвать полнофункциональное интернет-радио, вещающее в прямом эфире круглосуточно, с возможностью настраивать плейлисты и выбирать расписание для каждого из них. Помимо этого, имеется возможность вести прямые эфиры в режиме реального времени.


\nocite{*} % prints the entire bibliography without citing anything
\printbibliography[
  heading=bibintoc,
  title={СПИСОК ИСПОЛЬЗОВАННЫХ ИСТОЧНИКОВ},
]

\end{document}