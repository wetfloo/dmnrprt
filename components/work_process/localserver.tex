\section{Установка, настройка и тестирование интернет-радио на локальном сервере}

Следующим шагом после выбора программного обеспечения AzuraCast было изучать документации с официального сайта. Следуя официальному руководству, была установлена виртуальная машина Ubuntu for Windows и произведена установка AzuraCast на неё.

После перехода по стандартному IP - 127.0.0.1, мы попадаем на окно первоначальной настройки. \\
Создаём пользователя. \\
Создаём радиостанцию. \\
Вписываем URL-адрес сайта. \\
После этого мы попадаем на окно управления самой радиостанцией. Дальше переходим в медиафайлы. \\
Добавляем треки в хранилище сайта и добавляем их в плейлист default. \\
После этого мы можем запустить станцию и перейти по адресу 127.0.0.1/public/teststation, чтобы протестировать работу.