\section{Выбор програмного обеспечения для трансляции аудио}

Для создания интернет радиостанции были рассмотрены следующие варианты програмного обеспечения:

\begin{itemize}
  \item \textbf{Icecast}
        -- это сервер потокового мультимедиа (аудио/видео),  который в поддерживает потоки Ogg (Vorbis и Theora), Opus, WebM и MP3.

        Его можно использовать для создания интернет-радиостанции или частного музыкального автомата и многого другого. Он очень универсален в том смысле, что новые форматы могут быть добавлены относительно легко, и поддерживает открытые стандарты для общения и взаимодействия. Однако, его довольно трудно использовать сам по себе.
  \item \textbf{Media Server Control Panel}
        -- это самая передовая панель управления хостингом интернет-радио/телевидения с широким функционалом. Это автономное приложение, предоставляющее собственные веб- и FTP-сервисы. Однако, оно платно и имеет закрытый исходный код.
  \item \textbf{AzuraCast}
        -- это самодостаточный универсальный пакет для управления веб-радио с открытым исходным кодом. Он позволяет загружать мультимедиа, управляйтб плейлистами, создавать локальные точки монтирования и удалённые ретрансляторы, просматривать аналитику и отчёты из веб-браузера. Также AzuraCast легко устанавливается на сервер при помощи Docker-образа на сайте. Именно это ПО и было выбрано.
\end{itemize}
